\subsection{ARM Top-Level Software}
The ARM top-level software receives the image data from a remote device and sends the results back to this device. Control of the hardware.

\subsubsection{Requirements}

Requirements of the ARM Top-Level Software can be summarized as
\begin{itemize} 
	\item Receive image data
	\item Also use image data set already stored on device
	\item send results to remote PC
	\item Send and receive control signals from remote PC
	\item Send image data to driver user layer and receive results from driver user layer
	\item Send and receive status and control signals to driver user layer
	\item Run at start-up 
\end{itemize}

Because the 

\subsubsection{Dynamic Updating of the Bitstream}

This was initially planned but would required more time to finish.

Optional feature: Update Bitstream file using \texttt{/dev/xdevcfg}.

Update: For newer versions it looks like \texttt{/dev/xdevcfg} doesn't exist anymore. The problem is discussed here \footnote{\url{https://forum.digilentinc.com/topic/18194-dynamically-load-bitstream-on-petalinux/}} and a potential solution can be found here. \footnote{\url{https://github.com/Digilent/zynq-dynamic-tools}}

\subsubsection{Interface to remote PC}
See Section \ref{subsec:InterfaceRemoteZed}.  
\subsubsection{Interface to kernel layer}
Python wrapper are used for the interface between the top level software which is programmed in python and the hardware drivers which are programmed in C. For usability a high level interface to the underlying C wrapper is made. 
This header interface can then be wrapped to multiple target languages using \cite{Beazley:1996aa}. In our case this was done for Python and Numpy. 

